
% Section Muon Monitor
%
\section{Using Muons and Fast Photodetectors to Measure the Bunch-by-bunch Proton Intensity Profiles}
\label{muon_monitor}

The stroboscopic technique relies on the precise time relationship
between a given proton parent bunch interacting in the target and the `grandchild' neutrino
events detected in the neutrino detectors. We propose using fast
photodetectors to measure the intensity profile of the proton
interactions on a bunch-by-bunch basis with a resolution sufficient to
allow `time-slicing' of the neutrino vertices in the detector for
energy and flavor discrimination. For example, 10 psec resolution
on the profile of each proton bunch would introduce little smearing in 100 psec
time-slices at the neutrino detector.

%\subsection{Muon Monitor Systems}

The monitoring would be done with two conventional muon monitor
systems, one at 90 degrees to the target in the Lab frame (the
Transverse Muon Monitor System, or TMMS), and one
embedded in the shielding at the end of the decay volume (the Forward
Muon Monitor System, or FMMS). Each would
consist of photodetectors and waveform sampling electronics capable of a
time resolution on the order of 10 psec.


%\subsubsection{Transverse Muon Monitor System (TMMS)}
The transverse muon monitor system (TMMS) provides a bunch-by-bunch
profile of the proton interactions in the target. The TMMS consists of
a small area MCP-PMT Cherenkov telescope located on a vertical line to
the target, with substantial shielding between the target and the
counters. The telescope could be located above ground, or, as was done
in Ref.~\cite{E100}, located in a cladded hole drilled into the
ground, as determined by the rate and the desired momentum cut-off
provided by the muon range in the transverse shielding.

The TMMS views the proton bunches transversely, so that the time
offsets in the longitudinal direction do not introduce biases in the
proton bunch profile from energy-dependent effects. The flight path in
the transverse direction can be made to be short. The smearing in time of
hadron decay, even though independent of the longitudinal bunch
profile, would wash out the bunch structure. In addition to the short
decay path, the shielding in the
transverse direction will provide a lower momentum cutoff for muons, and
hence for parent hadrons, being determined by the minimum muon range
required to reach the TMMS.

%\subsubsection{Forward Muon Monitor System (FMMS)}
The forward muon monitor system (FMMS) provides a
bunch-by-bunch profile of muon neutrino production in the target. The
TMMS would consist of a large-area `range stack' of large-area MCP-PMT
photodetectors with precise time resolution, separated by shielding so
as to measure the energy spectrum of the muons by range. In addition,
each layer would consist of multiple \LAPPDTM modules in a transverse
pattern capable of measuring the directional information as well
as the time profile of the proton beam.

The FMMS directly records the muon time of arrivals in each
layer. The muon times are determined by the flight time of the parent
hadrons in the same way as the neutrino times measured at the
detector. As the mix of electron and muon neutrinos is energy and sign
dependent, the FMMS provides additional family information as well as
a stroboscopic measurement of the muon spectrum.

%\subsection{Photodetectors, Waveform Sampling, and Data Acquisition}

%
% Section: Synchronization of the Bunch Profile with the Neutrino
% Interaction Time
% 
%\subsection{Synchronization of the Bunch Profile with the Neutrino Interaction Time}
%\label{synchronization}
%\subsection{Time Stamps and Latency Budgets}
%\subsection{Local Clocks}
%\subsection{Synchronization of Local Clocks}
%\subsection{Monitoring and Verification}
The neutrinos arrive at both the near and far detectors before any
electronics signal can propagate, creating a substantial latency
between the proton bunch profile and the corresponding neutrino event profile.
Consequently at both the target and the detector there needs to be a
wave-form sampling and digital buffer with enough bandwidth to keep up
with the RF bunch frequency and enough depth to allow for the latency,
triggering, and readout. 

Several standard schemes employed at high
luminosity colliders and waveform sampling systems would mitigate high
data rates. Local processing in FPGAs located in the front-end
electronics allows sparcification and data compaction. Multiple buffering will allow deadtimeless
operation.  FPGAs also allow prescaling, with a small fraction of
bunches outputting the full sampling profile, with the bulk only
transmitting as many fitted moments as prove necessary to monitor
shapes over the time scale set by the prescale.

The synchronization of the neutrino event with its parent proton bunch
consequently is done with time stamps in both locations. This requires
a master clock available to both locations. One solution is two
synchronized atomic clocks, which have the requisite
accuracy~\cite{atomic_clocks}. 

Monitoring and verification of the synchronization is
necessary~\cite{speedy_italians}. Accumulated data at the `edges' of the
proton bunch train may verify that events only occur in occupied
bunches, registered to the correct timing.

