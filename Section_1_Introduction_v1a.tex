
\section{Introduction}
A deeper understanding of the neutrino sector, including CP violation,
the mass hierarchy, and deviations from unitarity of the PMNS matrix,
hinges on high precision, increasingly systematics-dominated
measurements of neutrino oscillation parameters. One approach to
reducing the systematic uncertainties is by expanding the range of L/E
and the mix of lepton family contributions measured simultaneously in
both near and far detectors. Here we present a scheme for such
measurements at Fermilab using the time-of-arrival of on-axis
neutrinos at the near and far detectors relative to the proton RF
structure.

The wide span of energies in neutrino beams stems from the wide range
of energies of the parent hadrons.  One technique for optimizing
the neutrino energy spectrum for measuring oscillations at a given
detector distance is to look at angles off-axis from the pointing of
the beam, a technique notably exploited by the NOvA~\cite{NOvAtdr} and T2K~\cite{T2Kconcept}
experiments.
% Matt is this correct?

An alternative method for understanding and selecting different energy
spectra within a neutrino beam exploits the differing velocities of
the parent hadrons. Lower energy pions and kaons travel more slowly,
especially as they approach sub-relativistic energies. As a
consequence, lower energy neutrinos are created further behind the rest of
the bunch. Selecting later arriving neutrinos would provide an
increasingly pure low-energy subset of the overall flux.\footnote{As shown
below, the time difference from hadron travel outweighs the compensating
effect that higher energy hadrons live longer.}

% Matt's sentences need referencing..
The idea of using neutrino arrival times to resolve kinematic details has a long
history. Efforts to detect dark matter have relied on time-of-flight
differences between dark matter particles and neutrinos. In 1998
M. Goldhaber pointed out that neutrinos from SN1987A were detected
earlier in Kamiokande than in IMB due to the correlation of energy and time of production~\cite{mgoldhaber}. The energy-dependent time
evolution of supernova neutrinos has also been proposed as a means of determining the mass hierarchy~\cite{Supernova_time_hierarchy}. In the context of neutrino beams, several notable efforts have utilized bunch timing to place limits on neutrino velocity~\cite{{OPERAspeed},{MINOSspeed},{T2Kspeed}}. MiniBooNE has already published several analyses exploiting timing to select stopped kaons from the NuMI beam~\cite{MiniBooNEstoppedK} and to search for heavy dark matter particles~\cite{MiniBooNEdm}.

The MiniBooNE collaboration has recently explored the idea of using
timing relative to the RF structure of the proton beam 
to select on the neutrino energy spectrum. However, efforts to
select different energy spectra on the basis of beam timing have been largely overlooked due to two considerations: (1) Limited time
resolutions of the detectors themselves were insufficient to see the
O(100) psec effect, and (2) the $\sim$1 nsec spread of the proton
bunch impinging on the target washes out most of the effect. We
address both these issues below.

Here we revisit the idea of using beam timing to select different
energy components of the neutrino flux, but with a higher-frequency RF
structure superimposed on the proton beam after normal acceleration
but before extraction to make shorter proton bunches.

%\subsection{Organizational Outline}
An overview of the `stroboscopic' approach introduced above is
presented in Section~\ref{approach}. Section~\ref{Fermilab} describes
the outlines of a first implementation at Fermilab.
Section~\ref{mechanism} describes the mechanism of energy and flavor
separation by `time slicing' relative to the parent proton bunch
structure, and the impact of proton bunch size on the separation.
Section~\ref{time_sorted_spectra} presents energy spectra for
electron, muon, and tau neutrinos and anti-neutrinos for time windows
referenced to the their identified parent proton bunch.

Section~\ref{RF} addresses the accelerator and RF issues of rebunching
the 120-GeV protons at higher RF frequency, starting with the 53-MHz bunch structure of the
Fermilab Main Injector. Bunch profiles are presented from simulations
of a 530-MHz Cornell-like SCRF cavity, ramped up after the 53-MHz rf voltage is ramped
down. Measuring the profile of each proton bunch relative to a precise
system clock using muons and a system of fast photodetectors is
described in Section~\ref{muon_monitor}.  Section~\ref{results}
summarizes the energy spectra for electron, muon, and tau neutrinos
and anti-neutrinos for the 530 MHz beam profiles of
Section~\ref{RF}. Conclusions and areas of needed development are
presented in Section~\ref{conclusions}.

