
%
% Section Conclusions
%
\section{Conclusions}
\label{conclusions}
A higher RF frequency than the current 53 MHz RF structure in the
Fermilab Main Injector (MI) would allow using precision time
measurements to statistically discriminate the energy and family of
neutrino events in on-axis detectors. For narrow proton bunches, the
time of the neutrino interaction in the near or far neutrino detector
relative to the production of the parent hadron in the target is
weighted towards lower energy for later events and higher energy for
earlier events. Selecting on time bins (a.k.a. the `stroboscopic'
technique) relative to the RF thus performs a similar sculpting of the spectrum for an on-axis geometry as going off-axis without losing the on-axis flux. The relative fractions of $e$, $\mu$, and $\tau$
neutrinos also change with time bin and with polarity. 

The technique relies on the implementation of 
precision time measurements at the neutrino
detector relative to those at the proton target, and may encourage 
renewed efforts implementing fast timing capability in LiA
detectors~\cite{LiA_timing}, and development of higher precision timing
photodetector systems at warm liquid-based detectors~\cite{ANNIE, JUNO,liquid_based}. Precision timing using psec
photo-detectors and improved event vertex reconstruction
algorithms~\cite{vertex_reconstruction} will be necessary.

We provide as an example a specific implementation of rebunching the
MI 53 MHz at flat-top with a commercially-available RF 
cavity~\cite{nsls-cavity,cls_stampe}. A simulation was made of the 53 MHz
ramp-down and 500 MHz ramp up in order to determine the final RMS
bunch width and the additional time in the cycle to re-bunch. 
Constraints on the implementation include aperture, beam loss to the
abort gap during the RF transition, and limiting the loss of
accumulated protons-on-target due to increased cycle time. The
implementation  appears feasible within these constraints with the 
single cavity.

The energy spectra of neutrinos in 50 (xxx) psec time slices relative
to the parent proton bunch at the target are presented over the range
0-xxx psec (see Section~\ref{results} for each neutrino family. The
later time slices select neutrinos with a lower energy spectrum, much
like an off-axis beam. The contribution of the different families also
changes with the energy selection, as expected. A small, but
interesting, contribution from tau neutrinos is enhanced in the `prompt' 
life-time bin.

% End of Section 9

